\documentclass[a4paper, 12pt,oneside]{scrbook}
\usepackage[english,ngerman]{babel} %mehrere Sprachen können eingebunden werden, letzt Sprache ist aktiv
\usepackage[utf8]{inputenc}
\usepackage[T1]{fontenc}
\usepackage{graphicx}
\usepackage{tabularx}
\usepackage{scrpage2}
%\usepackage{cite}
%\usepackage{natbib}
\usepackage{url}
\usepackage{amsmath}
%\bibliographystyle{abbrvnat}
%\setcitestyle{authoryear, open={(},close={)}}
\clearscrheadfoot
\begin{document}
	\frontmatter
	
\begin{titlepage}
	\thispagestyle{scrheadings}
	\ohead{\includegraphics[scale=1]{bilder/DHBW_Mosbach_Logo.jpg}}
	\ihead{\includegraphics[scale=0.2]{bilder/Weinig_Logo.png}}
	\centering	
	\vspace*{3cm}
	{\scshape\LARGE Firma Weinig AG\par}
	\vspace{1cm}
	{\scshape\LARGE Duale Hochschule Mosbach Baden-Württemberg\par}
	\vspace{1.5cm}
	{\large\bfseries Praxisbericht T1000\\ \huge Digitaltechnik\par}
	\vspace{2cm}
	{\Large\itshape Raphael Lawo\par}
	\vfill
	Ansprechpartner\par
	
	\vfill
	
	{\large\today\par}
\end{titlepage}
%\thispagestyle{empty}
\newpage
\setcounter{page}{1}
	\chapter*{\centering Zusammenfassung}\label{kap: abstract}


	\tableofcontents
	\listoffigures
	\listoftables
	\mainmatter
	\chapter{Einleitung}\label{kap: einleitung}
	\section{Firma Weinig AG}\label{kap: weinig}
	\section{Was ist Digitaltechnik im Allgemeinen}\label{kap: digitaltechnikAlg}
	\chapter{Digitaltechnik}\label{kap: digitaltechnik}
	\section{Zahlensysteme}\label{kap: zahlensysteme}
	-polyaedische Zahlensysteme
		-Dezimal
		-dual
		-hexa
	\subsection{Erklärung}\label{kap: erklärung}
	\subsection{Duales Zahlensystem}\label{kap: duales Zahlensystem}
	\section{Aussagenlogik/Schaltalgebra}\label{kap: aussagenlogik}
	\subsection{Erklärung}\label{kap: erklräungAussage}
	\subsection{Wahrheitstabelle}\label{kap: wahrheitstabelle}
	\subsection{Konjunktive Normalform}\label{kap: knf}
	\subsection{Karnaugh-Veitech-Diagramm}\label{kap: kv}
	\section{Logische Verknüpfungen}\label{kap: logischrVerknüpf}
	\subsection{Erklärung}\label{kap: erklärungLogisch}
	\subsection{AND}\label{kap: and}
	\subsection{OR}\label{kap: or}
	\subsection{XOR}\label{kap: xor}
	\subsection{NOT}\label{kap: not}
	\subsection{NAND}\label{kap: nand}
	\subsection{NOR}\label{kap: nor}
	\subsection{Vergleichsfunktionen}\label{kap: vergleichsfunk}
	\section{Speicherelemente}\label{kap: speicherelemente}
	\subsection{Erklärung}\label{kap: erklärungSpeicher}
	\subsection{SR-Element}\label{kap: srElement}
	\subsection{RS-Element}\label{key: rsElement}
	\chapter{SPS als zentrales Element in der Digitaltechnik}
	\subsection{Erklärung}\label{kap: erklärungSPS}
	\subsection{SPS-Adressierung}\label{kap: sps-adress}
	\subsection{Datentypen}\label{kap: datentypen}
	\subsection{Parametertypen}\label{kap: parametertypen}
	\subsection{Beispielprogramm in SPS}\label{kap: bspSPS}
	\subsubsection{Netzwerkstruktur erklären}\label{kap: netzwerk}
	\subsubsection{Erklärung des Ablaufes}\label{kap: ablauf}
	\subsubsection{Umsetzung in einer SPS}\label{kap: umsetzung}
	\backmatter
	

	
	
\end{document}